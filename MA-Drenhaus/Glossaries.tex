%************************************************
%Abkürzungen
%************************************************

\newcommand{\gr}[1]{\textbf{#1}}
\newcommand{\eng}[1]{\emph{#1}}
%Glossar Verweispfeil mit Text
\newcommand{\gloref}[1]{\glslink{#1}{\ding{43}\,Glossar} \glsadd{#1}}


\newacronym[description={Cyber-physisches System, engl. \eng{\gr{C}yber-\gr{P}hysical \gr{S}ystem}},%
	longplural={Cyber-physische Systeme, engl. \eng{\gr{C}yber-\gr{P}hysical \gr{S}ystems}}]%
	{cps}%
	{CPS}%
	{cyber-physisches System, engl. \eng{\gr{C}yber-\gr{P}hysical \gr{S}ystem}}%
	
\newacronym[description={Vielfachzugriff mit Trägerprüfung und Kollisionserkennung, engl. \eng{\gr{C}arrier-\gr{S}ense \gr{M}ultiple \gr{A}ccess / \gr{C}ollision \gr{A}voidance}}]%
	{csma}%
	{CSMA-CA}%
	{\eng{\gr{C}arrier \gr{S}ense \gr{M}ultiple \gr{A}ccess / \gr{C}ollision \gr{A}voidance}}%
	
\newacronym[description={Drahtloses Sensornetzwerk, engl. \eng{\gr{W}ireless \gr{S}ensor \gr{N}etwork}}]%
	{wsn}%
	{WSN}%
	{drahtloses Sensornetzwerk, engl. \eng{\gr{W}ireless \gr{S}ensor \gr{N}etwork}}%
	
\newacronym[description={Zugangspunkt, engl. \eng{\gr{A}ccess \gr{P}oint}}]%
	{ap}%
	{AP}%
	{\eng{\gr{A}ccess \gr{P}oint}}%
	
\newacronym[description={Medienzugriffssteuerung, engl. \eng{\gr{M}edia \gr{A}ccess \gr{C}ontrol}}]%
	{mac}%
	{MAC}%
	{Medienzugriffssteuerung, engl. \eng{\gr{M}edia \gr{A}ccess \gr{C}ontrol}}%
	
\newacronym[description={Engl. \eng{\gr{L}ogical \gr{L}ink \gr{C}ontrol}}]%
	{llc}%
	{LLC}%
	{\eng{\gr{L}ogical \gr{L}ink \gr{C}ontrol}}%
	
\newacronym[description={Engl. \eng{\gr{E}uropean \gr{T}elecommunications \gr{S}tandards \gr{I}nstitute}}]%
	{etsi}%
	{ETSI}%
	{\eng{\gr{E}uropean \gr{T}elecommunications \gr{S}tandards \gr{I}nstitute}}%
	
\newacronym[description={Auf Deutsch etwa \gqq{erst horchen, dann sprechen}, engl. \eng{\gr{L}isten \gr{B}efore \gr{T}alk}}]%
	{lbt}%
	{LBT}%
	{\eng{\gr{L}isten \gr{B}efore \gr{T}alk}}%
	
\newacronym[description={Engl. \eng{\gr{I}nstitute of \gr{E}lectrical and \gr{E}lectronics \gr{E}ngineers}}]%
	{ieee}%
	{IEEE}%
	{\eng{\gr{I}nstitute of \gr{E}lectrical and \gr{E}lectronics \gr{E}ngineers}}%

\newacronym[description={Kurzstreckenfunkgerät, engl. \eng{\gr{S}hort \gr{R}ange \gr{D}evice}}]%
	{srd}%
	{SRD}%
	{\eng{\gr{S}hort \gr{R}ange \gr{D}evice}}%

\newacronym[description={Offene Systemvernetzung, engl. \eng{\gr{O}pen \gr{S}ystems \gr{I}nterconnection}}]%
	{osi}%
	{OSI}%
	{\eng{\gr{O}pen \gr{S}ystems \gr{I}nterconnection}}%
	
\newacronym[description={Engl. \eng{\gr{I}nternational \gr{O}rganization for \gr{S}tandardization}}]%
	{iso}%
	{ISO}%
	{\eng{\gr{I}nternational \gr{O}rganization for \gr{S}tandardization}}%
	
\newacronym[description={Identifizierung durch elektromagnetische Wellen, engl. \eng{\gr{R}adio-\gr{F}requency \gr{I}dentification}}]%
	{rfid}%
	{RFID}%
	{Identifizierung durch elektromagnetischer Wellen, engl. \eng{\gr{R}adio-\gr{F}requency \gr{I}dentification}}%
	
\newacronym[description={Globale Handelsgut Indentifikationsnummer, engl.  \eng{\gr{G}lobal \gr{T}rade \gr{I}tem \gr{N}umber}}]%
	{gtin}%
	{GTIN}%
	{globale Handelsgut Indentifikationsnummer, engl. \eng{\gr{G}lobal \gr{T}rade \gr{I}tem \gr{N}umber}}%

\newacronym[description={Elektronischer Produkt Code, engl. \eng{\gr{E}lectronic \gr{P}roduct \gr{C}ode}}]%
	{epc}%
	{EPC}%
	{\eng{\gr{E}lectronic \gr{P}roduct \gr{C}ode}}%
	
\newacronym[description={Engl. \eng{\gr{O}bjective \gr{M}odular \gr{N}etwork \gr{T}estbed in C\gr{++}} \gloref{gl:omnet}}]%
	{omnet}%
	{OMNeT++}%
	{\eng{\gr{O}bjective \gr{M}odular \gr{N}etwork \gr{T}estbed in C\gr{++}}}%	

\newacronym[description={Drahtloses lokales Netzwerk, engl. \eng{\gr{W}ireless \gr{L}ocal \gr{A}rea \gr{N}etwork}}]%
	{wlan}%
	{WLAN}%
	{\eng{\gr{W}ireless \gr{L}ocal \gr{A}rea \gr{N}etwork}}%	
	
\newacronym[description={Drahtloses Netzwerk für das unmittelbare Umfeld, engl. \eng{\gr{W}ireless \gr{P}ersonal \gr{A}rea \gr{N}etwork}}]%
	{wpan}%
	{WPAN}%
	{\eng{\gr{W}ireless \gr{P}ersonal \gr{A}rea \gr{N}etwork}}%
	
\newacronym[description={IPv6 für \acs{wpan} mit geringer Leitungsaufnahme, engl. \eng{IPv\gr{6} over \gr{Lo}w Power \gr{WPAN}} \gloref{gl:6lowpan}}]%
	{6lowpan}%
	{6LoWPAN}%
	{\eng{IPv\gr{6} over \gr{Lo}w Power \gr{W}ireless \gr{P}ersonal \gr{A}rea \gr{N}etwork}}%		

\newacronym[description={Kanalbewertung, engl. \eng{\gr{C}lear \gr{C}annel \gr{A}ssessment}}]%
	{cca}%
	{CCA}%
	{\eng{\gr{C}lear \gr{C}annel \gr{A}ssessment}}%
	
\newacronym[description={Produktspezifische Zustellrate, engl. \eng{\gr{P}roduct \gr{S}pecific \gr{D}elivery} \gr{R}atio}]%
	{psdr}%
	{PSDR}%
	{produktspezifische Zustellrate engl. \eng{\gr{P}roduct \gr{S}pecific \gr{D}elivery} \gr{R}atio}%
	
\newacronym[description={Ereignisorientierte Simulation, engl. \eng{\gr{D}escrete \gr{E}vent \gr{S}imulation}},%
longplural={ereignisorientierten Simulationen, engl. \eng{\gr{D}escrete \gr{E}vent \gr{S}imulations}}]%]%
	{des}%
	{DES}%
	{ereignisorientierte Simulation, engl. \eng{\gr{D}escrete \gr{E}vent \gr{S}imulation}}%
	
\newacronym[description={Liste der zukünftigen Ereignisse bei \acsp{des}, engl. \eng{\gr{F}uture \gr{E}vent \gr{L}ist}}]%
	{fel}%
	{FEL}%
	{\eng{\gr{F}uture \gr{E}vent \gr{L}ist}}%	
	
	
\newacronym[description={Engl. \eng{\gr{I}ndustrial \gr{S}cientific \gr{M}edical} \gloref{gl:ism}}]%
	{ism}%
	{ISM}%
	{\eng{\gr{I}dustrial \gr{S}cientific \gr{M}edical}}%
	
\newacronym[description={Additives weißes gaußsches Rauschen, engl. \eng{\gr{A}dditive \gr{W}hite \gr{G}aussian \gr{N}oise}}]%
	{awgn}%
	{AWGN}%
	{additives weißes gaußsches Rauschen, engl.\eng{\gr{A}dditive \gr{W}hite \gr{G}aussian \gr{N}oise}}%
	
\newacronym[description={Signal-Rausch-Verhältnis, engl. \eng{\gr{S}ignal to \gr{N}oise \gr{R}atio}}]%
	{snr}%
	{SNR}%
	{Signal-Rausch-Verhältnis, engl. \eng{\gr{S}ignal to \gr{N}oise \gr{R}atio}}%
	
\newacronym[description={Frequenzumtastung, engl. \eng{\gr{F}requency \gr{S}hift \gr{K}eying}}]%
	{fsk}%
	{FSK}%
	{Frequenzumtastung, engl. \eng{\gr{F}requency \gr{S}hift \gr{K}eying}}%

\newacronym[description={Frequenzvielfachzugriff, engl. \eng{\gr{F}requency-\gr{D}ivision \gr{M}ultiple \gr{A}ccess}}]%
	{fdma}%
	{FDMA}%
	{Frequenzvielfachzugriff, engl. \eng{\gr{F}requency-\gr{D}ivision \gr{M}ultiple \gr{A}ccess}}%
	
\newacronym[description={Zeitvielfachzugriff, engl. \eng{\gr{T}ime-\gr{D}ivision \gr{M}ultiple \gr{A}ccess}}]%
	{tdma}%
	{TDMA}%
	{Zeitvielfachzugriff, engl. \eng{\gr{T}ime-\gr{D}ivision \gr{M}ultiple \gr{A}ccess}}%
	
\newacronym[description={Codevielfachzugriff, engl. \eng{\gr{C}ode-\gr{D}ivision \gr{M}ultiple \gr{A}ccess}}]%
	{cdma}%
	{CDMA}%
	{Codevielfachzugriff, engl. \eng{\gr{C}ode-\gr{D}ivision \gr{M}ultiple \gr{A}ccess}}%
	
\newacronym[description={Raumfachzugriff, engl. \eng{\gr{S}pace-\gr{D}ivision \gr{M}ultiple \gr{A}ccess}}]%
	{sdma}%
	{SDMA}%
	{Raumvielfachzugriff, engl. \eng{\gr{S}pace-\gr{D}ivision \gr{M}ultiple \gr{A}ccess}}%
	
\newacronym[description={Vielfachzugriff mit Trägerprüfung, engl. \eng{\gr{C}arrier-\gr{S}ense \gr{M}ultiple \gr{A}ccess}}]%
	{csma1}%
	{CSMA}%
	{Vielfachzugriff mit Trägerprüfung, engl. \eng{\gr{C}arrier-\gr{S}ense \gr{M}ultiple \gr{A}ccess}}%
	
\newacronym[description={Auf Deutsch \gqq{Anfrage zum Senden}, engl. \eng{\gr{r}equest \gr{t}o \gr{s}end}}]%
	{rts}%
	{RTS}%
	{\eng{\gr{r}equest \gr{t}o \gr{s}end}}%
	
\newacronym[description={Auf Deutsch \gqq{klar zum Senden}, engl. \eng{\gr{c}lear \gr{t}o \gr{s}end}}]%
	{cts}%
	{CTS}%
	{\eng{\gr{r}equest \gr{t}o \gr{s}end}}%	
	
\newacronym[description={Allgemeiner Ein- und Ausgabepin, engl. \eng{\gr{G}eneral \gr{P}urpose \gr{I}input \gr{O}utput}}]%
	{gpio}%
	{GPIO}%
	{\eng{\gr{G}eneral \gr{P}urpose \gr{I}nput/\gr{O}utput}}%	
	
\newacronym[description={Zentrale Recheneinheit, engl. \eng{\gr{C}entral \gr{P}rocessing \gr{U}nit}}]%
	{cpu}%
	{CPU}%
	{\eng{\gr{C}entral \gr{P}rocessing \gr{U}nit}}%	

\newacronym[description={Analog-Digital Wandler, engl. \eng{\gr{A}nalog to \gr{D}igital \gr{C}onverter}}]%
	{adc}%
	{ADC}%
	{Analog-Digital Wandler, engl. \eng{\gr{A}nalog to \gr{D}igital \gr{C}onverter}}%	
	
\newacronym[description={Rechner mit reduziertem Instruktionssatz, engl. \eng{\gr{R}educed to \gr{I}nstruction Set \gr{C}omputer}}]%
	{risc}%
	{RISC}%
	{\eng{\gr{R}educed to \gr{I}nstruction Set \gr{C}omputer}}%	
	
\newacronym[description={Ferroelektrischer Speicher mit wahlfreiem Zugriff, engl. \eng{\gr{F}erroelectric to \gr{R}andom \gr{A}ccess} \gr{M}emory}]%
	{fram}%
	{FRAM}%
	{\eng{\gr{F}erroelectric to \gr{R}andom \gr{A}ccess} \gr{M}emory}%	
	
\newacronym[description={Leuchtdiode, engl. \eng{\gr{L}ight \gr{E}mitting \gr{D}iode}}]%
	{led}%
	{LED}%
	{\eng{\gr{L}ight \gr{E}mitting \gr{D}iode}}%	
	
\newacronym[description={Engl. \eng{\gr{S}erial \gr{P}eripheral \gr{I}nterface} \gloref{gl:spi}}]%
	{spi}%
	{SPI}%
	{\eng{\gr{S}erial \gr{P}eripheral \gr{I}nterface}}%	

\newacronym[description={Engl. \eng{\gr{I}nterrupt \gr{S}ervice \gr{R}outine}}]%
	{isr}%
	{ISR}%
	{\eng{\gr{I}nterrupt \gr{S}ervice \gr{R}outine}}%	
	
\newacronym[description={Engl. \eng{\gr{M}aster \gr{O}ut \gr{S}lave \gr{I}n} \gloref{gl:mosi}}]%
	{mosi}%
	{MOSI}%
	{\eng{\gr{M}aster \gr{O}ut \gr{S}lave \gr{I}n}}%	
	
\newacronym[description={Engl. \eng{\gr{M}aster \gr{I}n \gr{S}lave \gr{O}ut} \gloref{gl:miso}}]%
	{miso}%
	{MISO}%
	{\eng{\gr{M}aster \gr{I}n \gr{S}lave \gr{O}ut}}%	
	
\newacronym[description={Engl. \eng{\gr{S}erial \gr{Cl}ock} \gloref{gl:scl}}]%
	{scl}%
	{SCL}%
	{\eng{\gr{S}erial \gr{Cl}ock}}%
	
\newacronym[description={Engl. \eng{\gr{C}hip \gr{S}elect}}]%
	{cs}%
	{CS}%
	{\eng{\gr{C}hip \gr{S}elect}}%

		
%************************************************
%Glossar
%************************************************

\newglossaryentry{industrie40}{%
	name={Industrie 4.0},%
	description={Beschreibt den zunehmenden Einfluss von Kommunikations- und Informationstechnik in klassischen Industrieabläufen. Grundlage dafür bilden intelligente, vernetzte Systeme}%
}

\newglossaryentry{node}{%
	name={Node},%
	description={Auf Deutsch \gqq{Knoten}. Umverteilungspunkt oder Endpunkt bei Datenübertragungen}%
}

\newglossaryentry{802154}{%
	name={IEEE 802.15.4},%
	description={Standardisierter Protokollstapel für drahtlose Netzwerke mit geringer Datenrate}%
}

\newglossaryentry{80211g}{%
	name={IEEE 802.11g},%
	description={Standardisierter Protokollstapel für den Informationsaustausch in \acp{wlan}}%
}

\newglossaryentry{transceiver}{%
	name={Transceiver},%
	description={Funkgerät mit integrierter Sende- und Empfangseinheit, engl. \eng{\gr{Trans}mitter Re\gr{ceiver}}}%
}

\newglossaryentry{gl:omnet}{%
	name={OMNeT++},%
	description={\gls{cpp} Framework für ereignisorientierte Simulationen, speziell für Netzwerkprotokolle (\acl{omnet})}%
}

\newglossaryentry{cpp}{%
	name={C++},%
	description={Erweiterung der Programmiersprache \gls{c}. Erlaubt objektorientiertes Programmieren}%
}

\newglossaryentry{c}{%
	name={C},%
	description={Imperative Programmiersprache, die vorwiegend zur maschinennahen Programmierung verwendet wird}%
}

\newglossaryentry{inet}{%
	name={INET},%
	description={Modellbibliothek für \gls{gl:omnet} für leitungsgebundene, kabellose und mobile Netzwerke}%
}

\newglossaryentry{bluetooth}{%
	name={Bluetooth},%
	description={Funktechnologie zum Datenaustausch zwischen Geräten über kurze Distanzen im $2,4 GHz$ Band}%
}

\newglossaryentry{gl:6lowpan}{%
	name={6LoWPAN},%
	description={Implementierung des Internet Protokolls für \acsp{wpan} mit geringer Leitungsaufnahme (\acs{6lowpan}) und damit Voraussetzung für den Zugang zum Internet}%
}
 
\newglossaryentry{routing}{%
	name={Routing},%
	description={Ermitteln eines geeigneten Weges einer Datenübertragung in einem Netzwerk}%
}

\newglossaryentry{energyharvesting}{%
	name={Energy Harvesting},%
	description={Auf Deutsch \gqq{Energie ernten}. Bezeichnet die Gewinnung elektrischer Energie aus der Umgebung durch Licht, Bewegung oder Temperaturunterschiede}%
}

\newglossaryentry{aloha}{%
	name={ALOHA},%
	description={Kanalzugriffsverfahren, das 1970 von Norman Abramson an der Universität von Hawaii entwickelt wurde \citep{aloha}}%
}

\newglossaryentry{server}{%
	name={Server},%
	description={Computer innerhalb eines Netzwerks, der anderen Computern oder Geräten, den \glspl{client}, Dienste zur Verfügung stellt}%
}

\newglossaryentry{client}{%
	name={Client},%
	description={Computer oder Gerät, das Dienste von einem \gls{server} in einem Netzwerk anfordert}%
}

\newglossaryentry{zigbee}{%
	name={ZigBee},%
	description={Standardisierter Protokollstapel für Netzwerke mit geringem Datenaufkommen. ZigBee basiert auf dem \gls{802154} Standard und erweitert diesen um eine Netzwerk- und Anwendungsschicht }%
}

\newglossaryentry{gl:ism}{%
	name={ISM-Band},%
	description={Frequenzbereiche für die Nutzung in Industrie, Wissenschaft und Medizin, eng. \acf{ism}}%
}

\newglossaryentry{hiddenstation}{%
	name={Hidden Station},%
	description={Auf Deutsch \gqq{verborgene Station}. Beschreibt die Unfähigkeit, die Funkaktivität weiter entfernter Sender zu detektieren. Siehe auch \autoref{kap:grundlagen_sec:hiddenstation}}%
}

\newglossaryentry{cc1200}{%
	name={CC1200},%
	description={Funkchip für den Sub-GHz Bereich von Texas Instruments. Siehe auch \autoref{kap:grundlagen_sec:cc1200}}%
}

\newglossaryentry{msp430}{%
	name={MSP430},%
	description={Mikrocontroller-Familie von Texas Instruments speziell für Anwendungen mit geringer Leistungsaufnahme. Die genaue Typenbezeichnung des verwendeten Modells lautet MSP430FR9596. Siehe auch \autoref{kap:grundlagen_sec:msp430}}%
}

\newglossaryentry{launchpad}{%
	name={LaunchPad},%
	description={Entwicklungsplatine für den \gls{msp430} Mikrocontroller}%
}

\newglossaryentry{gl:gpio}{%
	name={GPIO},%
	description={Anschlusspin an Prozessoren oder Mikrocontrollern zur allgemeinen Datenein- und Ausgabe, engl. \acl{gpio}}%
}

\newglossaryentry{watchdog}{%
	name={Watchdog},%
	description={Auf Deutsch \gqq{Wachhund}. Komponente eines Mikrocontrollers zum kontrollierten Neustart. Siehe \autoref{kap:grundlagen_sec:msp430}}%
}

\newglossaryentry{gl:spi}{%
	name={SPI},%
	description={Synchrones serielles Bussystem zum Datenaustausch zwischen einem \emph{Master}-Gerät und einem oder mehreren \emph{Slave}-Geräten. Siehe \autoref{kap:grundlagen_sec:msp430}}%
}

\newglossaryentry{gl:mosi}{%
	name={MOSI Leitung},%
	description={Datenleitung beim \gls{gl:spi} für die Datenübertragung vom \emph{Master} zum \emph{Slave}, engl. \acf{mosi}. Siehe \autoref{kap:grundlagen_sec:msp430}}%
}

\newglossaryentry{gl:miso}{%
	name={MISO Leitung},%
	description={Datenleitung beim \gls{gl:spi} für die Datenübertragung vom \emph{Slave} zum \emph{Master}, engl. \acf{miso}. Siehe \autoref{kap:grundlagen_sec:msp430}}%
}

\newglossaryentry{gl:scl}{%
	name={SCL Leitung},%
	description={Taktleitung zur synchronen Datenübertragung beim \gls{gl:spi}, engl. \acf{scl}. Siehe \autoref{kap:grundlagen_sec:msp430}}%
}

\newglossaryentry{gl:cs}{%
	name={CS Leitung},%
	description={Siganlleitung zur auswahl des \emph{Slaves} beim \gls{gl:spi}, engl. \acf{cs}. Siehe \autoref{kap:grundlagen_sec:msp430}}%
}


 
%*****************************************
%*****************************************
%*****************************************
%*****************************************
%*****************************************




